
\documentclass[10pt,a4paper]{article}
    \usepackage[utf8]{inputenc}
    \usepackage{xcolor}
    % \usepackage[utf8x]{inputenc} % para poder usar tildes en archivos UTF-8
    \usepackage[spanish,es-tabla]{babel}
    \usepackage{verbatim}
    \usepackage{clrscode3e}
    \usepackage{amssymb}
    \usepackage{amsmath}
    \usepackage{amsfonts} % for the \checkmark command 
    \usepackage{graphicx}
    \usepackage{float}
    \usepackage{pdfpages}
    \usepackage{enumerate}
    \usepackage{caption}
    \usepackage{subcaption}
    \usepackage[left=3cm, right=3cm, top=2cm]{geometry}
% \documentclass[10pt,a4paper]{article}
% \usepackage[utf8x]{inputenc} % para poder usar tildes en archivos UTF-8
% \usepackage[spanish,es-tabla]{babel}
% \usepackage{verbatim}
% \usepackage{amsmath}
% \usepackage{clrscode3e}
% \usepackage{amssymb}
% \usepackage{graphicx}
% \usepackage{float}
% \usepackage{pdfpages}

%\usepackage{bibtex}

%\usepackage{a4wide} % márgenes un poco más anchos que lo usual

\usepackage{caratula} % Se puede descargar en ~> https://github.com/bcardiff/dc-tex
\usepackage[breaklinks=true]{hyperref}


\begin{document} % Todo lo que escribamos a partir de aca va a aparecer en el documento.

%fran
%\sloppy

% Completar los datos de la caratula
\titulo{Trabajo Práctico 2}
\fecha{\today}
\materia{Algoritmos y Estructuras Datos III}
\grupo{Grupo ``Apruebennos que Santi es de Racing''}

% Completar los integrantes del grupo:)
\integrante{Cristian Kubrak}{456/15}{kubrakcristian@gmail.com}
\integrante{Santiago Feliu}{644/15}{santiagofeliu@gmail.com}
\integrante{Pablo Ingaramo}{544/15}{pablo2martin@hotmail.com}
\integrante{Kennedy William Rios Cuba}{503/15}{wrios@dc.uba.ar}

\maketitle 

% \par \textbf{Abstract:} El objetivo de este trabajo es resolver el problema de la suma de
% subconjuntos mediante 3 diferentes t\'ecnicas: Fuerza Bruta, Backtracking y 
% Programaci\'on din\'amica para luego analizar cu\'al de ellas resulta mejor aplicar seg\'un el contexto.

\tableofcontents
% \par  \textbf{Palabras clave:} subset sum - brute force - Backtracking - Programaci\'on din\'amica
% Aca comienzan a escribir su informe


\newpage

\section{Introducción}
% \input{introduccion}
\newpage

\section{Desarrollo}
% \input{desarrollo}
\newpage

\section{Experimentaci\'on}
% \input{experimentacion}
\newpage

\section{Resultados}
% \input{resultados}
\newpage

\section{Conclusiones}
% \input{conclusiones}
\newpage

\section{Ap\'endice}
% \input{apendice}


\end{document}

