\par Comenzamos con la matriz de adyacencia $(W)_{ij}$ que luego transformamos en $(W')_{ij}$ donde $w'_{ij} = - \log w_{ij}$
\par Tal como mencionamos previamente, creamos esta nueva matriz para que poder encontrar una soluci\'on al problema
del arbitraje se limite a hallar ciclos negativos.
\par No solo nos importa decidir si existe o no un ciclo negativo, sino que tambi\'en nos interesa poder reconstruirlo
(ese ser\'a la sucesi\'on de compra/venta de divisas que deber\'iamos realizar para obtener gananacia). 
Para esto utilizamos variantes de dos algoritmos.
\subsubsection{Floyd-Warshall}
Para encontrar un ciclo negativo, utilizamos en escencia el algoritmo de Floyd-Warshall solo que agregando algunas
cosas:
\begin{enumerate}
\item Contamos tambi\'en con una matriz de $nxn$ en el cual guardamos los nodos siguientes al que estamos recorriendo.
\item A la hora de actualizar la matriz de distancias, tambi\'en actualizamos la de nodos siguientes.
\item Una vez finalizadas el algoritmo de FW, revisamos si alg\'un elemento en la diagonal de distancias es negativo
(esto implica que existe un ciclo negativo).
\end{enumerate}
Sea $i$ el nodo en el cual encontramos un ciclo negativo ($distancias[i][i] < 0$)

\begin{codebox}
    \Procname {\proc{CicloNegativo}(Matriz($int$) $siguientes$, int $i$)}
    \li $recorrido$.agregar($i$)
    \li $v = i$
    \li $u$ = $siguientes[u][v]$
    \li \While ($u \neq v$)
        \Then
    \li     $recorrido$.agregar($u$)
    \li     $u$ = $siguiente[u][v]$
    \End
    \li $recorrido$.agregar($i$)
    \li \Return $recorrido$
\end{codebox}

\par A la hora de analizar la complejidad de esta variaci\'on de Floyd-Warshall partimos de la base que FW tiene una complejidad
de $O(n^3)$. Nuestra modificaci\'on solo agrega la creaci\'on de la matriz $siguientes$ ($O(n^2)$) y luego actualizarla 
cada vez que FW actualiza distancias lo cual no afecta la complejidad asint\'otica del algoritmo.
\par Para reconstruir el ciclo negativo, el mismo tendr\'a longitud a lo sumo $n$ raz\'on por la cual la guarda de 
la linea 4 se ejecutar\'a a lo sumo $n$ veces.
\par En conclusi\'on la ejecuci\'on de este algoritmo tendr\'a como complejidad $(O(n^3))$

\subsubsection{Bellman-Ford}
Para encontrar un ciclo negativo, utilizamos en escencia el algoritmo de Bellman-Ford introduciendo algunas
modificaciones para poder reconstruir el ciclo negativo:

\begin{enumerate}
\item Contamos tambi\'en con un vector $predecesor$ en el cual almacenamos el nodo predecesor al que
estamos investigando.
\end{enumerate}
